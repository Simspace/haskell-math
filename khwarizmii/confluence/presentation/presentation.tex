% -*- mode: latex-mode; -*-

\documentclass{beamer}

%%%%%%%%%%%%%%%%%%%%%%%%%%%%%%%%%%%%%%%%%%%%%%%%%%%%%%%%%%%
% Dark Beamer Theme (Heavily Customized)                  %
% Source: https://github.com/pblottiere/dark-beamer-theme %
%%%%%%%%%%%%%%%%%%%%%%%%%%%%%%%%%%%%%%%%%%%%%%%%%%%%%%%%%%%

\usetheme{dbt}

%%%%%%%%%%%%%%%%%%%%%%%%%%%%%%%%%%%%%%%%%%%%
% TikZiT WYSIWYG Editor                    %
% Source: https://tikzit.github.io/#custom %
%%%%%%%%%%%%%%%%%%%%%%%%%%%%%%%%%%%%%%%%%%%%

\usepackage{tikzit}
\begin{tikzpicture}
	\begin{pgfonlayer}{nodelayer}
		\node [style=none] (0) at (0, 3) {$x$};
		\node [style=none] (1) at (-3, 0) {$y$};
		\node [style=none] (2) at (0, -3) {$\exists p$};
		\node [style=none] (3) at (3, 0) {$z$};
	\end{pgfonlayer}
	\begin{pgfonlayer}{edgelayer}
		\draw [style=transitive edge] (0.center) to (3.center);
		\draw [style=transitive edge] (0.center) to (1.center);
		\draw [style=transitive edge] (1.center) to (2.center);
		\draw [style=transitive edge] (3.center) to (2.center);
	\end{pgfonlayer}
\end{tikzpicture}


%%%%%%%%%%%%%%%%%%%%%%%%%%%%%%%%%%%%%%%%%%%%%%%%%
% Source: https://tex.stackexchange.com/a/74127 %
%%%%%%%%%%%%%%%%%%%%%%%%%%%%%%%%%%%%%%%%%%%%%%%%%


\usepackage{tcolorbox}

\tcbset{%
  colback=black!95,%
  colupper=white,%
  colframe=white!40!black,%
  sharp corners%
}

%%%%%%%%%%%%%%%%%%%%%%%%%%%%%
% Deduction Rule Formatting %
%%%%%%%%%%%%%%%%%%%%%%%%%%%%%

\usepackage{mathpartir}

%%%%%%%%%%%%%%%%%%%%%%%%%%%%%%%%%%%%%%%%%%%%%%%%%%%%%%%%%%%%%%%%%%%%%%
% Source: https://isabelle.in.tum.de/community/Generate_TeX_Snippets %
%%%%%%%%%%%%%%%%%%%%%%%%%%%%%%%%%%%%%%%%%%%%%%%%%%%%%%%%%%%%%%%%%%%%%%

\usepackage{generated/document/comment}%
\usepackage{generated/document/isabelle}%
\usepackage{generated/document/isabellesym}%
\usepackage{generated/document/isabelletags}%
\usepackage{ifthen}%

\newcommand{\DefineSnippet}[2]{%
  \expandafter\newcommand\csname snippet--#1\endcsname{ \kern-1ex #2
    \kern-1.5ex }}

\newcommand{\Snippet}[1]{%
  \ifcsname snippet--#1\endcsname {\csname snippet--#1\endcsname}
  \else +++++++ERROR: Snippet ``#1'' not defined+++++++ \fi}

\input{generated/snippets}

% https://www.cl.cam.ac.uk/research/hvg/Isabelle/dist/Isabelle2017/doc/sugar.pdf
\usepackage{mathpartir}

%%%%%%%%
% Main %
%%%%%%%%

\title{Lambda-Calculus Confluence} \author{Matthew Doty} \date{}

\begin{document}

{\setbeamertemplate{footline}{} \frame{\titlepage}}


\hypersetup{colorlinks=false}

\section*{Outline}%

\hypersetup{colorlinks}

\begin{frame}[shrink=35]{\insertsectionhead}%
  \ \
  \tableofcontents %
\end{frame}

\sectionframe{Why Confluence?}

\subsection{Stuck Macros}
\begin{frame}[plain, allowframebreaks]{\insertsectionhead\ \textemdash\
    \insertsubsectionhead}
  \includegraphics[width=\linewidth]{figs/macro1.pdf}

  \framebreak

  \includegraphics[width=\linewidth]{figs/macro2.pdf}
\end{frame}

\subsection{Na\"ive Typed Macros Aren't Confluent}
\begin{frame}[plain]{\insertsectionhead\ \textemdash\\
    \insertsubsectionhead}
  \begin{flushright}
  \includegraphics[width=0.35\linewidth]{figs/macro4.pdf}
  \end{flushright}

  \includegraphics[width=\linewidth]{figs/macro3.pdf}
\end{frame}

\sectionframe{Untyped Lambda Calculus}

\subsection{Syntax}
\begin{frame}[fragile, allowframebreaks]{\insertsectionhead\ \textemdash\
    \insertsubsectionhead}
\begin{code}
module LambdaCalc where
import Prelude hiding (foldl, foldl')

infixl 7 :.

data Lam
  = V Int
  | Lam :. Lam
  | Abs Int Lam
  deriving Eq
\end{code}

  \framebreak
    \begin{table}[htbp]
      \begin{tabular}{ll}
        Conventional & Haskell \\
        \hline
        & \\
        $\lambda x.\ x$ & |Abs 1 (V 1)| \\
        $\lambda z.\ z$ & |Abs 2 (V 2)| \\
        $\lambda x. \lambda y.\ x$ & |Abs 1 (Abs 2 (V 1))| \\
        $(\lambda x. \lambda x.\ x)\ (\lambda y.\ y)$ & |Abs 1 (Abs 1 (V 1)) :. Abs 2 (V 2)|
      \end{tabular}
    \end{table}


\end{frame}

\subsection{Variable Capture}
\begin{frame}[fragile, allowframebreaks]{\insertsectionhead\ \textemdash\
    \insertsubsectionhead}
\begin{code}
freeVars :: Lam -> [Int]

freeVars (V x) = [x]

freeVars (lam1 :. lam2) =
  freeVars lam1 <> freeVars lam2

freeVars (Abs x lam) =
  [y | y <- freeVars lam, y /= x]
\end{code}

  \framebreak

\begin{code}
freshVar :: Lam -> Int

freshVar lam = 1 + foldr max 0 (freeVars lam)
\end{code}
\end{frame}

\subsection{Substitution}
\begin{frame}[fragile, allowframebreaks]{\insertsectionhead\ \textemdash\
    \insertsubsectionhead}

A single variable substitution:

\begin{code}
infix 6 :=

data Subst = Int := Lam
\end{code}~\\

Substitution operation:

\begin{code}
infixl 5 !

(!) :: Lam -> Subst -> Lam
\end{code}

\framebreak

Conventional \LaTeX
  \[
    s\; [x\ :=\ t]
  \]

  Our Haskell Implementation
  \begin{center}
  |s ! x := t|
  \end{center}

\framebreak

\begin{code}
V y ! x := t
  | x == y = t
  | otherwise = V y
\end{code}

\framebreak

\begin{code}
lam1 :. lam2 ! x := t =
  (lam1 ! x := t) :. (lam2 ! x := t)
\end{code}

\framebreak

\begin{code}
s@(Abs y lam) ! x := t
  | x == y = Abs y lam
  | x /= y && not (y `elem` freeVars t) =
    Abs y (lam ! x := t)
  | otherwise =
    Abs z (lam ! y := V z ! x := t)
  where
    z = freshVar (s :. t :. V x)
\end{code}
\end{frame}

\subsection{Beta Rule}
\begin{frame}[fragile]{\insertsectionhead\ \textemdash\
    \insertsubsectionhead}

Conventional \LaTeX
  \[
    (\lambda x.\, s)\; t\ \to_\beta\ s\; [x\ :=\ t]
  \]~\\
  Haskell implementation
  \begin{center}
     |(Abs x s) :. t| $\ \to_\beta\ $ |s ! x := t|
  \end{center}
\end{frame}

\subsection{Evaluation Strategies}
\begin{frame}[fragile, allowframebreaks]{\insertsectionhead\ \textemdash\
    \insertsubsectionhead}
\begin{code}
normEval :: Lam -> Lam

normEval ((Abs x s) :. t) =
  normEval (s ! x := t)

---
normEval (V x) = V x

normEval (s :. t) =
  (normEval s) :. (normEval t)

normEval (Abs x s) = Abs x (normEval s)
\end{code}

\framebreak

\begin{code}
callByVal :: Lam -> Lam

-- normEval ((Abs x s) :. t) =
--   normEval (s ! x := t)

callByVal ((Abs x s) :. t) =
  callByVal (s ! x := callByVal t)

---
callByVal (V x) = V x

callByVal (s :. t) =
  (callByVal s) :. (callByVal t)

callByVal (Abs x s) = Abs x (callByVal s)
\end{code}

\framebreak

\begin{code}
appEval :: Lam -> Lam

-- callByVal ((Abs x s) :. t) =
--   callByVal (s ! x := callByVal t)

appEval ((Abs x s) :. t) =
  appEval (appEval s ! x := t)

---
appEval (V x) = V x

appEval (s :. t) =
  (appEval s) :. (appEval t)

appEval (Abs x s) = Abs x (appEval s)
\end{code}

\framebreak

\begin{code}
lazyEval :: Lam -> Lam

-- normEval (Abs x s) = Abs x (normEval s)
lazyEval (Abs x s) = Abs x s

---
lazyEval ((Abs x s) :. t) =
  lazyEval (s ! x := t)

lazyEval (V x) = V x

lazyEval (s :. t) =
  (lazyEval s) :. (lazyEval t)
\end{code}

\end{frame}

\subsection{Digression: Weak Head Normal Form}
\begin{frame}[fragile, allowframebreaks]{\insertsectionhead\ \textemdash\\
    \insertsubsectionhead}

\begin{code}
weakHeadNF :: Lam -> Bool

weakHeadNF (V _) = True

weakHeadNF ((Abs _ _) :. _) = False

weakHeadNF (s :. t) =
  weakHeadNF s && weakHeadNF t

weakHeadNF (Abs _ _) = True
\end{code}

\framebreak

\begin{claim}
\begin{center}
  if \\~\\
  |weakHeadNF x| \\~\\
  then \\~\\
  |lazyEval x == x|
\end{center}
\end{claim}

\framebreak

\begin{code}
foldl :: (a -> b -> a) -> a -> [b] -> a
foldl _f a [] = a
foldl f a (x:xs) = foldl f (f a x) xs
\end{code}

\framebreak

\begin{lstlisting}
foldl (+) 0 [1, 2, 3, 4]
  = foldl (0 + 1) [2, 3, 4]
  = foldl ((0 + 1) + 2) [3, 4]
  = foldl (((0 + 1) + 2) + 3) [4]
  = foldl ((((0 + 1) + 2) + 3) + 4) []
  = ((((0 + 1) + 2) + 3) + 4)
  = (((1 + 2) + 3) + 4)
  = ((3 + 3) + 4)
  = 6 + 4
  = 10
\end{lstlisting}

\framebreak

Haskell has a special primitive |seq :: a -> b -> b| \\~\\

When Haskell tries to evaluate

\begin{center}
  |x `seq` y|
\end{center}

Haskell will first evaluate |x| to weak head normal form, and then
evaluate |y| to weak head normal form

\framebreak

\begin{code}
foldl' :: (a -> b -> a) -> a -> [b] -> a
foldl' f a [] = a
foldl' f a (x : xs) =
  let a' = f a x
   in a' `seq` foldl' f a' xs
\end{code}

\framebreak

\begin{lstlisting}
foldl' (+) 0 [1, 2, 3, 4]
  = foldl' (+) 0 [1, 2, 3, 4]
  = let a' = 0 + 1
    in a' `seq` foldl' (+) a' [2, 3, 4]
  = 1 `seq` foldl' (+) 1 [2, 3, 4]
  = foldl' (+) 1 [2, 3, 4]
\end{lstlisting}

\framebreak

\begin{lstlisting}
foldl' (+) 0 [1, 2, 3, 4]
  = foldl' (+) 1 [2, 3, 4]
  = foldl' (+) 3 [3, 4]
  = foldl' (+) 6 [4]
  = foldl' (+) 10 []
  = 10
\end{lstlisting}

\end{frame}

\subsection{Normal Form}
\begin{frame}[fragile, allowframebreaks]{\insertsectionhead\ \textemdash\\
    \insertsubsectionhead}

\begin{code}
nf :: Lam -> Bool

-- weakHeadNF (Abs _ _) = True
nf (Abs _ s) = nf s

---
nf (V _) = True

nf ((Abs _ _) :. _) = False

nf (s :. t) = nf s && nf t
\end{code}

\framebreak
\begin{claim}
  If |nf x| then
  \begin{itemize}
    \item |normEval x == x|
    \item |callByVal x == x|
    \item |appEval x == x|
  \end{itemize}
\end{claim}
\end{frame}

\sectionframe{Church's Arithmetic}

\begin{subsectionframe}{Church Numerals}{plain}
  \begin{table}
    \begin{tabular}{ll}
      Number & Church Numeral \\
      \hline
      & \\
      0 & $\lambda f. \lambda x.\ x$ \\
      1 & $\lambda f. \lambda x.\ f\ x$ \\
      2 & $\lambda f. \lambda x.\ f\ (f\ x)$ \\
      3 & $\lambda f. \lambda x.\ f\ (f\ (f\ x))$ \\
      \vdots & \vdots \\
      n & $\lambda f. \lambda x.\ f^{\; \circ n}\ x$
    \end{tabular}
  \end{table}
\end{subsectionframe}

\begin{subsectionframe}{Arithmetical Operations}{plain, allowframebreaks}
  \begin{eqnarray*}
    (+) & \equiv & \lambda m. \lambda n. \lambda f. \lambda x.\ m\ f\ (n\ f\ x)
  \end{eqnarray*}
  \ldots since $f^{\; \circ (m + n)} x = f^{\; \circ m}\ (f^{\; \circ n}\ x)$
  \framebreak
  \begin{eqnarray*}
    succ & \equiv & \lambda n. \lambda f. \lambda x.\ f\ (n\ f\ x) \\
         & \approx_\beta &  (+ 1)
  \end{eqnarray*}
  \framebreak
  \begin{eqnarray*}
    (\ast) & \equiv & \lambda m. \lambda n. \lambda f. \lambda x.\ m\ (n\ f)\ x\\
  \end{eqnarray*}
  \ldots since $f^{\; \circ (m * n)}\ x = (f^{\; \circ n})^{\circ m}\ x$
\end{subsectionframe}

\begin{subsectionframe}{Is Church's Arithmetic Consistent?}{plain}
  \emph{Can we make an arithmetical expression $e$ which evaluates to $1$ using one strategy, but $2$ using some other strategy?}
  \\~\\
  (kind of a ``philosophy of math'' version of Sam G's problem)
\end{subsectionframe}

\sectionframe{Formalization: de Bruijn Notation}

\begin{subsectionframe}{Syntax}{plain, allowframebreaks}
  \begin{table}
    \begin{tabular}{ll}
      Conventional & de Bruijn \\
      \hline
      & \\
      $\lambda x.\ x$ & $\mathbf{\lambda}\ 0$ \\
      $\lambda z.\ z$ & $\mathbf{\lambda}\ 0$ \\
      $\lambda x. \lambda y.\ x$ & $\mathbf{\lambda} \mathbf{\lambda}\ 1$\\
      $(\lambda x. \lambda x.\ x)\ (\lambda y.\ y)$ & $(\lambda \lambda\ 0)\ (\lambda\ 0)$\\~\\
      $(\lambda x.\ x\ x)\ (\lambda x.\ x\ x)$ & $ (\lambda \lambda\ 0\ 0)\ (\lambda \lambda\ 0\ 0)$ \\
      $\lambda x. \lambda y. \lambda s. \lambda z.\ x\ s\ (y\ s\ z)$ & $\mathbf{\lambda} \mathbf{\lambda} \mathbf{\lambda} \mathbf{\lambda}\ 3\ 1\ (2\ 1\ 0)$ \\
    \end{tabular}
  \end{table}
  \framebreak
  \begin{table}
    \begin{tabular}{lll}
      Number & Church Numeral & de Bruijn \\
      \hline
      & & \\
      0 & $\lambda f. \lambda x.\ x$ & $\mathbf{\lambda} \mathbf{\lambda}\ 0$ \\
      1 & $\lambda f. \lambda x.\ f\ x$ & $\mathbf{\lambda} \mathbf{\lambda}\ 1\ 0$ \\
      2 & $\lambda f. \lambda x.\ f\ (f\ x)$ & $\mathbf{\lambda} \mathbf{\lambda}\ 1\ (1\ 0)$ \\
      3 & $\lambda f. \lambda x.\ f\ (f\ (f\ x))$ & $\mathbf{\lambda} \mathbf{\lambda}\ 1\ (1\ (1\ 0))$ \\
    \end{tabular}
  \end{table}
  \framebreak
  \Snippet{db_Syntax}
\end{subsectionframe}

\begin{subsectionframe}{Lifting}{plain}
  \Snippet{lifting_def}
\end{subsectionframe}

\begin{subsectionframe}{Substitution}{plain}
  \Snippet{substitution_def}
\end{subsectionframe}

\subsection{Beta Reduction}
\begin{frame}[fragile, allowframebreaks]{\insertsectionhead\ \textemdash\
    \insertsubsectionhead}
  \begin{center}
  \Snippet{beta_beta}
  \end{center}

  ~\\~\\

  \hspace*{\fill}
  \Snippet{beta_appL}
  \hfill
  \Snippet{beta_appR}
  \hspace*{\fill}

  ~\\~\\

  \begin{center}
  \Snippet{beta_abs}
\end{center}

\framebreak

$\beta$-rule in Isabelle/HOL

  \begin{center}
\Snippet{beta_beta_no_rule}
  \end{center}

$\beta$-rule for the Haskell implementation

  \begin{center}
     |(Abs x s) :. t| $\ \to_\beta\ $ |s ! x := t|
  \end{center}

\end{frame}


\subsection{Beta equivalence}
\begin{frame}[fragile, allowframebreaks]{\insertsectionhead\ \textemdash\
    \insertsubsectionhead}

  \hspace*{\fill}
  \Snippet{beta_equiv_refl}
  \hfill
  \Snippet{beta_equiv_subst}
  \hspace*{\fill}

  ~\\~\\

  \hspace*{\fill}
  \Snippet{beta_equiv_abs}
  \hfill
  \Snippet{beta_equiv_symm}
  \hspace*{\fill}

  ~\\~\\

  \begin{center}
  \Snippet{beta_equiv_app}
\end{center}

\framebreak

\begin{lemma}
  \begin{center}
  \Snippet{beta_equiv_alt_def}
  \end{center}
\end{lemma}

\framebreak

Chains of reductions like this:
  \begin{center}
    \tikzfig{figs/transitive_reduction}
  \end{center}

  \framebreak

Can be drawn like this:

  \begin{center}
    \tikzfig{figs/transitive_reduction2}
  \end{center}

  \vspace{-50pt} \ldots and are represented symbolically with $\to^\ast$\\~\\

\framebreak
\begin{center}
  \Snippet{beta_equiv}
~\\~\\
\begin{center}
  is graphically depicted by
\end{center}
~\\~\\
\tikzfig{figs/church-rosser}
\end{center}

\end{frame}

\begin{subsectionframe}{Consistency Redux}{plain}
  Stronger consistency challenge:\\~\\
  \begin{center}
    Can we show $1 \not\approx_\beta 2$?
  \end{center}
\end{subsectionframe}

\sectionframe{Abstract Confluence}

\subsection{Definition of Confluence}
\begin{frame}[fragile]{\insertsectionhead\
    \textemdash\ \insertsubsectionhead}
A relation $\to$ is \emph{confluent} when for all $x$, $y$ and $z$ where
$x \to^\ast y$ and $x \to^\ast z$, there exists a (not necessarily
unique) $p$ where $y \to^\ast p$ and $z \to^\ast p$

  \begin{center}
      \tikzfig{figs/confluence}
  \end{center}
\end{frame}

\subsection{The Church-Rosser Property}
\begin{frame}[fragile, allowframebreaks]{\insertsectionhead\
    \textemdash\ \insertsubsectionhead}
  Define the predicate
  \begin{center}
    \Snippet{Church_Rosser_def_lhs}
  \end{center}
  to mean
  \begin{center}
        \Snippet{Church_Rosser_def_rhs}
  \end{center}
  \framebreak
  \begin{center}
    \tikzfig{figs/church-rosser}~\\~\\
    if and only if\\~\\
    \tikzfig{figs/church-rosser-2}
  \end{center}
\end{frame}

\subsection{Alternate Definition of Confluence}
\begin{frame}[plain]{\insertsectionhead\
    \textemdash\ \insertsubsectionhead}
  \begin{theorem}
    \begin{center}
      \Snippet{Church_Rosser_confluent}
    \end{center}
    A relation $R$ is confluent if and only if $R$ has the \emph{Church-Rosser} property.
  \end{theorem}
\end{frame}

\subsection{Diamond Property}
\begin{frame}[fragile, allowframebreaks]{\insertsectionhead\
    \textemdash\ \insertsubsectionhead}
  A relation $\to$ has \emph{the diamond property} if for all $x$, $y$, and $z$ if $x \to y$ and $x \to z$ then there is a (not necessarily unique) $p$ such that $y \to p$ and $z \to p$

\framebreak

  \begin{center}
      \tikzfig{figs/diamond}
  \end{center}

\framebreak

\begin{lemma}
  \begin{center}
  \Snippet{confluent_diamond}
  \end{center}
\end{lemma}

\framebreak

\begin{theorem}
  \begin{center}
  \Snippet{diamond_confluent}
  \end{center}
\end{theorem}

\framebreak

\begin{claim}
  The untyped $\lambda$-calculus does not have the diamond property
  \end{claim}

  To see this, consider:
  \[
  (\lambda y.\ u\ y\ y)\ ((\lambda x.\ x)\ z)
  \]

\end{frame}


\sectionframe{Confluence of Beta-Reduction}

\subsection{Parallel Reduction}
\begin{frame}[fragile]{\insertsectionhead\
    \textemdash\ \insertsubsectionhead}
  Attributed to William Tait and Per Martin L\"of \cite{takahashiParallelReductionsLCalculus1995} \\~\\

  \begin{center}
  \Snippet{par_beta_var}
  \end{center}

  ~\\~\\

  \hspace*{\fill}
  \Snippet{par_beta_abs}
  \hfill
  \Snippet{par_beta_app}
  \hspace*{\fill}

  ~\\~\\

  \begin{center}
  \Snippet{par_beta_beta}
  \end{center}


\end{frame}

\subsection{Squeeze Theorems}
\begin{frame}[plain]{\insertsectionhead\
    \textemdash\ \insertsubsectionhead}
  \begin{center}
      \Snippet{beta_implies_par_beta}
  \end{center}
~\\
  \begin{center}
    \Snippet{par_beta_implies_transitive_beta}
  \end{center}
\end{frame}

\subsection{Confluence}
\begin{frame}[fragile, allowframebreaks]{\insertsectionhead\
    \textemdash\ \insertsubsectionhead}
  \begin{lemma}
    \begin{center}
      \Snippet{diamond_par_beta}
    \end{center}
  \end{lemma}
  ~\\
  \begin{lemma}
    \begin{center}
      \Snippet{par_beta_confluent}
    \end{center}
  \end{lemma}
  \framebreak
  \begin{theorem}
    \begin{center}
      \Snippet{beta_confluent}
    \end{center}
  \end{theorem}
  \framebreak
  \begin{claim}
    \[ 1 \not\approx_\beta 2 \]
  \end{claim}
\end{frame}


\section{Bibliography}
\begin{frame}{\insertsectionhead}
  \setbeamerfont{bibliography item}{size=\tiny}
  \setbeamerfont{bibliography entry author}{size=\tiny}
  \setbeamerfont{bibliography entry title}{size=\tiny}
  \setbeamerfont{bibliography entry location}{size=\tiny}
  \setbeamerfont{bibliography entry note}{size=\tiny}
  \bibliographystyle{siam}%
  \bibliography{bibliography}
\end{frame}

\end{document}
